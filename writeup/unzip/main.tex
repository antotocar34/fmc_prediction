\documentclass[twocolumn,10pt]{article}
\usepackage[letterpaper,left=1.95cm,right=1.95cm,top=2.5cm,bottom=3.16cm]{geometry} % Márgenes y demás
\usepackage{amsmath}
\usepackage{bm}
\usepackage{natbib}
\usepackage{bookmark}
\usepackage{float}
\usepackage{booktabs}
\bibliographystyle{apalike}
\usepackage[utf8]{inputenc}
\usepackage{pxfonts}
\usepackage{wasysym} % symbols
\usepackage{amssymb} % symbols
\usepackage{abstract}
\usepackage{parskip}
\usepackage{hyperref}
%\def\sym##1{\ifmmode^{##1}\else\(^{##1}\)\fi}
\usepackage{titlesec}
\usepackage{xcolor}
\usepackage{hyperref}
\hypersetup
{
    colorlinks = false,
    linkbordercolor = {white}, citebordercolor = {white}
}
\usepackage{float}
\usepackage{graphicx}
\usepackage{blindtext}% only for dummy text
\usepackage{fancyhdr}
\usepackage{titling}
\usepackage{caption}
\captionsetup[table]{
  position=above,
  justification=raggedright,
  labelsep=newline, % <<< label and text on different lines
  singlelinecheck=false % <<< raggadright also when the caption is shorter
                        % than a single line
}
\usepackage[document]{ragged2e}
\usepackage[affil-sl]{authblk} 
% \usepackage{mathptmx}  %Times new Roman

\setlength{\columnsep}{1cm}
\renewcommand{\refname}{References}
\renewcommand{\figurename}{Fig.}
\renewcommand{\tablename}{Table}
\renewcommand{\abstractname}{ABSTRACT}


%%%%%%%% FORMATO DE LAS SECCIONES Y SUBSECCIONES %%%%%
\titleformat{\section} 
	{\normalfont\large\bfseries}{\makebox[30pt][l]{\thesection}}{0pt}{} 
\titleformat{\subsection} 
	{\normalfont\large\bfseries }{\makebox[30pt][l]{\thesubsection}}{0pt}{}
\renewcommand\thesection{\arabic{section}.}
\renewcommand\thesubsection{\thesection\arabic{subsection}}
%%%%%%%% FORMATO DE LAS SECCIONES Y SUBSECCIONES %%%%%


	
\makeatletter
\long\def\@makecaption#1#2{
\vskip\abovecaptionskip
\sbox\@tempboxa{#1. #2}
\ifdim \wd\@tempboxa >\hsize
#1. #2\par
\else
\global \@minipagefalse
\hb@xt@\hsize{\hfil\box\@tempboxa\hfil}
\fi
\vskip\belowcaptionskip}
\makeatother


\pagestyle{fancyplain}% <- pagestyle fancyplain
\fancyfoot[C]{}
\renewcommand\plainheadrulewidth{.4pt}% headrule on plain pages
%\fancyhf{}
\rhead{}
\lhead{Statistical Modelling and Inference}
\rfoot{\thepage}





%%%%%% CORRESPONSE AUTHOR %%%%%
\fancypagestyle{alim}
{
\fancyfoot[C]{}
\renewcommand\plainheadrulewidth{.4pt}% headrule on plain pages
%\fancyhf{}
\fancyhf{}\renewcommand{\headrulewidth}{0.4pt}
\fancyhead[R]{DECIDE ON PAGE HEADER HERE}
\lhead{Statistical Modelling and Inference }
    \renewcommand\plainheadrulewidth{.4pt}% headrule on plain pages
}

%%%%%%%%%%%%%%%%%%%%%%%%%%%%%%%%%%

\date{}
  %%Llama el archivo de los estilos

%%%%%%% ENCABEZADO %%%%%%%%%%%%%%%%%%%%

{\title{\fontsize{16}{16}
\vspace*{-0.7cm}
\textbf{
Prediction of Live Fuel Moisture Content in Southern California
}[0.2cm] \\
}
}

\author{\fontsize{10pt}{10pt}\selectfont \textbf{Antoine Carnec and Clinton Leung}}


%%%%%%%%%%%%%%%%%%%%%%%%%%%%%%%%% INICIO DEL DOCUMENTO
%%%%%%%%%%%%%%%%%%%%%%%%%%%%%%%%%%%%%%%%%%%%%%%%%%%%%%%%%%%%%%%%%%%%%%%


\begin{document}
\thispagestyle{alim}

\justifying

TODO LIST

1) Update References to Bibtex throughout document \\
2) Finish Related Work section \\
3) Finish Data Processing sections: LANDSAT, DATA PROCESSING (Include plots of distribution of species and potentially the distribution of LFMC over the data if it is instructive), Also adding footnotes to indicate symbol representation of all predictors \\
4) Complete BMS and BMA sections (could potentially be retitled) \\
5) Add Abstract if we have room. Amend Discussion, Related Work and Introduction if necessary \\
6) Edit for typos, redundant information, coherence, consistency, continuity. \\
7) Generate summary tables to input to Appendix (Append Github link as well, assuming this is sufficient) \\
8) Reformat/typeset tables, figures, appendix  \\
9) Make title page



\section{INTRODUCTION - 0.5 Page}

As the world has increasingly seen the rise of wildfires due to climate change, it is an important topic to try and improve predictive models for the ignition, spread, and severity of wildfires. Research has shown that the fuel moisture content of vegetation is an important determining factor in the ignition, behaviour and severity of wildfires. The fuel moisture content of vegetation can be split up into two categories: live fuel moisture content (LFMC) and dead fuel moisture content (DFMC). Both have been found to have crucial effects on the spread of wildfires. There are many models that predict DFMC well because of its static nature; on the other hand LFMC deals with living organisms and the moisture content of such vegetation is much more dynamic and harder to model. The available models that are used to predict LFMC are often found to be lacking. Furthermore, LFMC observations are painstakingly collected by hand. Not only does it require the researcher to travel into remote areas---often the most important areas in terms of wildfire risk---but they must also go through the long process of oven drying the samples. Moreover, this sampling cannot practically be used to create real-time predictions during fire season which rely on predictive models that in turn rely on being trained on these manual samples of LFMC. \\

The most prevalent models used for prediction of LFMC employ the use of meteorological indices and drought indices. It is found that these methods lead to somewhat useful results on a large scale, but when considered under a species or smaller scale spatial level, the models begin to fall apart with very high spatial variability. % TODO add citation here
More recent approaches have begun to employ the use of remote sensing data with machine learning techniques to generate real-time predictions on a country level. These results are promising, but are once again susceptible to high spatial variability. Some studies have indicated that this spatial variability could be attributed to the fact that different species of vegetation react differently to the same weather conditions; some species can be considered high-responding, whose moisture content responds strongly to a change in weather, whereas others are low-responding. Research indicates that there is promise in using more mechanistic indices, especially as data on vegetation and vegetation distribution becomes better. \\

Our study investigates the viability of this mechanistic approach by combining meteorological factors, plant characteristics and remote sensing data. We focus our attention on the region of Southern California, an area of high wildfire risk, to prototype a model of this sort. Our model predicts the LFMC of nine different plant species scattered across Southern California using meteorological predictors (temperature, precipitation, relative humidity, solar radiation, and wind speed), NDVI, and various plant traits (specific leaf area, nitrogen per dry mass of leaf, phosphorous per dry mass of leaf, and plant height). A successful model would still depend on strong knowledge of the distribution of the vegetation within the area of concern, but this is outside the scope of this study. Apart from that, all of the predictors used are either readily available through automatic reporting or features of vegetation that are common domain knowledge a priori, which conforms to the constraints of creating a model for real-time applications. \\

\section{RELATED WORK - 0.25 Page}

In \citep{ruffault2018}, the authors assess and show that the prevailing method of using drought indices to predict LFMC is plagued by its limited ability for spatial predictability and cannot provide reliable estimates on a local level. The main benefit of using drought indices is that the data can be automatically sensed and is relatively widely available. An improvement on this method, as suggested by the study is to use a more mechanistic approach.
% TODO Add citation here
 The authors in \citep{castro2003} illustrates that when considering a specific plant species, meteorological predictors are sufficient to be able to produce strong predictions (approximately 0.8 $R^{2}$) which are able to generalize to different locations and maintain its predictive ability. The limitation here is that if one seeks to create a model that could be generalized for large areas, one would need to get accurate estimates of the distribution of vegetative species. A helpful approach to ameliorate the uncertainty in the distribution of species is to take advantage of latent traits amongst species that could give us information on species we have not even sampled. For example, \citep{denison2009} points out that plants can be largely categorized into high and low responding categories indicating their moisture content responsiveness to drought. \\

Another new approach to predicting LFMC is a more deterministic approach that uses NDVI as a proxy for the moisture content. Furthermore, this can be combined with land surface temperature measurements which can also be inferred from remote sensing satellite data. This approach is detailed in \citep{chuvieco2004} using a physics based approach involving microwave backscatter. They are quite successful in obtaining strong predictions ($R^{2}$ of about 0.8 using just multiple linear regression. The model was based off of a similar dataset of LFMC samples in the Mediterranean, but are only specifically successful with a small number of shrub and grassland species. Similarly, \citep{mccandless2020} uses remote sensing data, but with more flexible models on the WFAS dataset for the entire United States. The goal of this study was to create a real-time remote-sensing predictive model at the national level. This study primarily uses land surface temperature and satellite bands (which are used to derive NDVI) as predictors, and multiple linear regression, neural networks, random forest, and gradient boosted regression as models. It achieves an overall mean squared absolute error of about 20\% which is approximately 25\% of the standard deviation of LFMC, but there is no evaluation of the accuracy of the model at a small scale. Furthermore, the computational costs involved were significant, involving over 4 TB of data and the use of the US NCAR Casper cluster supercomputer. \\

These approaches have either been able to create large scale predictions that suffer from spatial variability at a small scale, or have been able to create strong predictions locally that do not necessarily generalize across all dimensions of the problem. \\

Main articles:

1) Enhancing Wildfire Spread Modelling by Building a Gridded Fuel moisture Content product with machine Learning \\
Key points:\\
Data used: WFAS, MODIS Sattelite for bands 1-7, WRF-Hydro data for soil saturation, evapo, geographical characteristics \\
Models used: GBR, ANN, Random Forest, Multiple Linear Regression \\
Shortcomings: Large amounts of data 4TB. Elevation was the predictor of greatest importance. Mean absolute error of 21.92 on a mean vallue of 94.8 with SD 86.2. High spatial variability potentially not captured. \\

2) Combining NDVI and surface temperature for the estimation of live fuel moisture content in forest fire danger rating \\
methods: focused on areas of grasslands and shrubs \\
data: NDVI and LST, AVHRR\\
models: multiple linear regression \\
results: strong results of 0.85 +, but remember only for grasslands and shrubs. Also showed that LST was just a strong predictor for grasslands/shrubs (check on this) \\


3) How well do meteorological drought indices predict live fuel moisture content (LFMC)? An assessment for wildfire research and operations in Mediterranean ecosystems \\
4) Modelling moisture content in shrubs to predict wildfire risk in Catalonia (Spain) \\
Methods: Hand sampling over 2 years. \\
Models: Multiple linear backwards regression and GLM. used site as fixed factor \\
Shortcomings: Only applicable to their two specific species of shrubs. \\
Pros: good results with $R^2$ of about $0.8$ and captures spatial variability for those specific species \\

4b) Critical live fuel moisture in chaparral ecosystems: A threshold for fire activity
and its relationship to antecedent precipitation\\
Discusses the importance of LFMC predictions for Sourthern California, especially the 79 percent threshold\\

5) Continental-scale prediction of live fuel moisture content using soil moisture information\\ 
6) Evaluating remotely sensed live fuel moisture estimations for fire behaviour predictions in Georgia, USA

\section{DATASET - 1 Page}
In this study, we brought together four different datasets to predict the live fuel moisture content (LFMC). The LFMC observations come from surface observations sampled manually. The meteorological data comes from automatic surface weather stations. The satellite data comes from the Landsat7 satellite. The plant characteristics come from the TRY plant database which is a curated conglomeration of many databases that largely involve manual sampling. \\

\subsection{WFAS}
Observations of LFMC content are provided by the USFS Wildfire Assessment System. This contains a national database of LFMC as well as dead fuel moisture content (DFMC) samples. LFMC is given by \[\frac{Water Weight}{Oven Dried Weight} \cdot 100\] so it is possible for these values to be higher than 100. This study focuses on the Southern California subset of this data which contains a total of 37,912  total observations spanning 1982-05-06 to 2021-11-02. Although there are strong guidelines for the sampling of observations, the sampling is largely carried out by citizen scientists or volunteers. This means the observations may not be completely reliable, as well as the information on site location is approximate at best. We used reverse geocoding to find the nearest coordinate values that the site names may indicate. \\

\subsection{Meteorological Data}
Meteorological data from automatic weather stations is often not consistent, and can contain many errors. Furthermore, databases such NOAA's Climate Data Online archives actually have large numbers of missing observations and many gaps and inconsistencies. One option was to use paid weather data services that generate interpolated data to fill in the gaps due to instrument errors, but this does not make sense for a study whose ultimate goal is real time prediction. Finally, we were able to use the Mesonet provided by Synoptic which as a third party gathers and quality controls weather stations across the USA. The most reliable network for our region was the California Irrigation Management Information System. From this network we were restricted to retrieving only the most common weather variables that would assuredly be available in all of our stations. Furthermore, we only used weather stations that were at most 30km away from our geocoded WFAS site locations. The combination of this reduced our dataset to 7628 observations that we would consider. From the weather stations we used precipitation, wind speed, solar radiation, relative humidity, and temperature. The quality controlled weather stations only had a small number of missing observations as well as observations that were clear measurement errors\footnote{Anything outside of ranges: $0< Solar Radiation < 1100w/m^2$; $0 < Precipitation < 100mm$; $0 < Wind Speed < 50km/hr$; $0 < Relative Humidity < 100$; $-15< Temp < 55C$ daily were considered as measurement errors based on the annual records of the area. }. For these missing variables we made a local imputation by taking the average of the five preceding observations under the assumption that the weather is most similar to what had just passed. All of these readings were provided hourly. Using these readings we could extracted features pertaining to the three, seven and fifteen day rolling average, maximum, and minimums for each of these five initial parameters\footnote{}. \\

\subsection{TRY Plant Database}

Vegetation specific characteristics were retrieved from the TRY Plant Trait Database which is a conglomeration of many datasets worldwide curated by a network of vegetation scientists. From this database we requested a long list of traits for our species of concern and trimmed this list down to four traits for relevance and data availability: nitrogen content per dry leaf area, phosphorous content per dry leaf area, plant height, and specific leaf area (a ratio of leaf area to leaf biomass). These characteristics were chosen for their relevance in conjunction with NDVI data under the hypothesis that different plants have different relationships between NDVI and moisture content, and this can be characterized by the properties of their leaves as well as their overall plant height which is not easily inferred by a satellite scan of NDVI\footnote{}. \\

\subsection{Landsat 7}

The Landsat 7 is a satellite that was put into orbit in 1999 with a 16 day orbit cycle. It detects 8 spectral bands from earth's surface.
We collected data at each site location from the satellite for the range of dates available to us from the WFAS dataset. % TODO change wording
We calculate the Normalized Vegetation Index (NDVI), which is a measure of the density of live vegetation on land---NDVI is a common predictor of LFMC in the literature.
To keep things simple we took the mean NDVI in a 10km radius around the associated site of each LFMC observation.
% TODO mention why this may be a bad idea.

\subsection{Data Processing}
%A we were able to generate drought indices `that`
From the meteorological data, we were able to generate drought indices that have been widely used to predict fuel moisture content and in wildfire predictive indices such as the Canadian Forest Fire Weather Index. These include the Duff Moisture Code (DMC), the Drought Code (DC), and the Build up Index (BUI)\footnote{See appendix for relevant formulas: DMC(formulas \ref{eqn:dmc1}-\ref{eqn:dmc6}), DC(formulas \ref{eqn:DC1}-\ref{eqn:dc4}), and BUI(formula \ref{eqn:bui})}. We also generated a growth cycle variable based on the day of the year to capture the intrinsic plant cycle throughout the seasons based on the formula 
\[
    D = \cos\left(\frac{2\pi D}{365} - 0.59\right)
\]
from \citep{castro2003}.  The initial meteorological features were further aggregated to produce for each variable (precipitation, humidity, temperature, wind speed, and solar radiation) their respective minimum, maximum and mean for the previous three, seven and fifteen days. Finally, we created interactions between the growth cycle and the meteorological variables to capture the different responses of plants to weather based on the phase of their growth cycle.\\

Due to the timing of satellite overflights from Landsat 7 (one observation every 16 days), NDVI data was not available on a daily basis. To adjust for NDVI not matching the target observation date we used an exponential decay function
\begin{equation}
    \label{eqn:downweight}
    ANDVI_{t} = \rho^{-h} \cdot NDVI_{t-h}\\
\end{equation}
to downweight the importance of each observation. We set $\rho=1.8$ to quickly decay in importance in the first 30 days. We then created interactions between our adjusted NDVI (ANDVI) and our vegetation characteristic variables to capture the hypothesis that ANDVI given the characteristics that generate ANDVI can be a proxy for fuel moisture content rather than just ANDVI alone.
Furthermore, it is key to note that NDVI and NDVI related measures are included for their predictive value, but from an inference perspective it can only be viewed as a proxy of some latent unobservable characteristic of individual plant species that would also determine their LFMC.\\

% TODO we need a justification for this
We removed species Red Shank, Eastwoods Manzanita, and Brittlebrush from our study because these groups contained fewer than 25 observations each. \\

\subsection{Data Exploration}% TODO DO THIS FUCKING SECTION

Our data contains $n=6498$ observations and $p = 122$ variables, 
starting in late 2003 to November 2021.
Table \ref{tab:samp} outlines the number of variables per Fuel and per stations.

The distribution of the LFMC variable is shown in figure X.

\begin{table}{}
\centering
\begin{tabular}{l c}
\toprule
\textbf{Plant Type} & \textbf{N = 6,498}\textsuperscript{1} \\ 
\hline \\
buckwheat, eastern mojave & 205 (3.2\%) \\ 
ceanothus, bigpod & 155 (2.4\%) \\ 
ceanothus, hoaryleaf & 130 (2.0\%) \\ 
chamise & 3,243 (50\%) \\ 
chamise, new growth & 1,071 (16\%) \\ 
sage, black & 557 (8.6\%) \\ 
sage, purple & 162 (2.5\%) \\ 
sagebrush, black & 421 (6.5\%) \\ 
sagebrush, california & 554 (8.5\%) \\ 
 \bottomrule
\end{tabular}
\caption{Sample distribution across Fuels.}
\label{tab:samp}
\end{table}

% \begin{figure}[htpb]
%     \centering
%     {\includegraphics[width=0.8\textwidth]{/home/carneca/Documents/Latex/images/}}
%     \caption{}
%     \label{}
% \end{figure}

\section{REGRESSION - 4 Pages AND SUPERVISED/UNSUPERVISED LEARNING ~ 4 Pages}

% TODO Write this out in a better way
We estimate the following basic regression model.
\[
y = \bm{\beta}^{T}\bm{X} + \varepsilon, \quad \varepsilon \sim \mathcal{N}(0, \rho I)
\]
We will analyse this model mainly using Bayesian methods. We avoid using LASSO or other frequentist methods as
we have many highly correlated variables. % TODO Find a nice justification here

Since we are estimating a Bayesian model, we must specify the priors we are placing on our parameters.
We let
\begin{align*}
    \bm{\beta} &\sim \mathcal{N}(0, ng \rho (\bm{X}^{T}\bm{X})^{-1}) \\
    \rho &\sim IG(0.1, 0.1)
\end{align*}
which is a standard choice of priors for Bayesian regression.

We set $g$ to a minimally reflect a minimally informative prior.

\subsection{Bayesian Model Selection and Model Averaging}
% 1) Selection of best model and bayesian model averaging to estimate vs OLS vs Lasso on the completely pooled model \\
% Related figs: \\
% Related tables: Outputs of OLS, BMA, Lasso\\

% TODO Discuss basic theory
We run a Bayesian Model Selection procedure, which samples from a posterior distribution of a \emph{spike and slab} prior $\bm{\gamma} \in \{0,1\}^{p}$.
The idea is that the $\bm{\gamma}$ variable indicates which covariates are included in the model, i.e\;\;$\beta_j \neq 0$ if $\gamma_j = 1$ . So if we set a prior on $\bm{\gamma}$ we can apply Bayes theorem and sample from $p(\bm{\gamma} \;|\; \mathbf{y})$, which allows us to find which models are most probable under the data.
We set $\bm{\gamma} \sim \text{BetaBinomial(1,1)}$, which sets a uniform prior on model size. % TODO Justify this

First, we can examine the marginal posterior probability that a variable is included in the model $p(\gamma_j = 1 \;|\; y)$.


% Table \ref{tab:unpooled} shows the results for running the model on table 6.
% TODO Add figures for convergence checks or at least mention that there are convergence checks available.



\subsection{Bayesian Model Averaging}
% 2) Use BMS for completely unpooled to estimate coefficients instead of LASSO (due to collinearity concerns, an the ability to provide inference) and OLS(lack of shrinkage, high number of parameters). Use inference on coefficients to inform mixed effects model considering the bayesian model average estimates for the coefficients and their confidence intervals. \\
% Related figs: Maybe a figure with the most important variable coefficients plotted for each group in a different colour?\\
% Related tables : Choice of important variables for each group with coefficients/CI's \\

We use Bayesian Model Averaging (BMA) to compute point estimates for our $\beta$ coefficients. 
This gives the expected value of a coefficient over all possible models.
\[
    E\left[\bm{\bm{\beta}} \;|\; \bm{y}\right] = \sum_{\gamma_j}^{} E\left[\beta_j \;|\; \gamma_j = 1, \bm{y}\right] p(\gamma_j = 1 \;|\; \bm{y}).
\]

It is worth noting that we are not too interested in the actual point estimates of the model. Since our focus is mostly on prediction, and the problem
we are studying is much too complex to assume a simple causal structure. 
To naively interpret the coefficients as the marginal effect of a covariate on LFMC could be misleading, since we have not accounted for the effect of confounding variables---see \citep{westreich2013} for a discussion.
We present the BMA point estimates for the whole unpooled model in table \ref{table:unpooled} for those variables with a high marginal probability, i.e those variables with a high probability of being contained in the \emph{true} model. % TODO Is this correct?

The important weather variables like average solar radiation and minimum temperature are significant when interacted with the seasonality variable $D$. This reflects the non-linear (in this case sinusoidal) relationship in weather variables. 

We can see that the interaction variables containing the plant traits and NDVI are likely to be in the true model.
This highlights the relationship between plant traits and the vegetation denseness.
For example, the interaction between plant height and NDVI is around $-4$. This indicates that the influence of NDVI on LFMC is dependent on plant height. For smaller plants, low NDVI (coarsely, low green-ness) influences LFMC less than for taller plants.
This lends credence to our approach of using plant traits to model LFMC. 

% TODO 
% TODO Is significant the right word here?
From our grouped fuel analysis, we analyse the difference in Bayesian estimates across plants.
That is we run a model selection for each plant,
the motivation is that we are trying to isolate common significant variables common to plants
in order to motivate our hierarchical model design in section \ref{sec:hierarchical}.

\begin{figure}[htpb]
    \centering
    {\includegraphics[width=0.5\textwidth, scale=0.5]{./figures/bms/comparison_plot_old.png}}
    \caption{Comparison of BMA for regressions subsetted by Plant.
             Variables shown are those for which the unpooled BMS placed a high probability on.}
    \label{fig:comparison}
\end{figure}

Figure \ref{fig:comparison} shows the BMA estimate for those variables who were probably in the unpooled model.

We can see that there is variation between the plants..
For example, the Drought Code is an important predictor 
Of course it is not clear from this that the variables differ
because of plant characteristics or because of location characteristics; since one plant is mostly located in one location. % TODO replace this in the data.


\subsection{Principal Components Analysis}
Many of the meteorological variables are collinear with each other as is the nature of weather variables. We use PCA to reduce the dimensionality of the problem under the assumption that there are some smaller number of latent signals that drive weather patterns. Looking at the scree plot in figure \ref{fig:screeplot} we can see that a significant amount of variation is described in the first component with the largest gap between component one and two. From this plot it appears that if there existed truly a latent variable signal structure, then it would consist of the first 4 components. Table \ref{tab:pca_components} outline the makeup of the first six principal components. It is instructive to see that the principal components group our variables into intuitive clusters that could represent the true signals that are driving the variation in our data. These signals for the first six principal components can be largely categorized as seasonal temperature changes, relative humidity, wind speed, accumulated precipitation, NDVI's interaction with plant characteristics, and solar radiation. 
\begin{table}[!htbp] \centering 
\begin{tabular}{p{18mm}|p{58mm}} 
\hline 
\hline \\ 
PC1: Temperature & $D_{t}$, $D_{t}:T_{max15}$, $D_{t}:T_{mean15}$, $D_{t}:T_{max7}$, $D_{t}:T_{mean7}$, $D_{t}:T_{max3}$, $D_{t}:T_{mean3}$, $D_{t}:T$, $D_{t}:G_{max15}$ \\%, $D_{t} G_{max7}$, $D_{t} T_{max}$, $D_{t} G_{max3}$, $D_{t} T_{min15}$, $E_{mean15}$, $D_{t} T_{min7}$, $D_{t} G_{mean15}$, $D_{t} G_{max}$, $D_{t} G_{mean7}$, $D_{t} T_{min3}$, $D_{t} W_{mean15}$ \\
PC2: Humidity & $H_{mean7}$, $H_{mean15}$, $H_{mean3}$, $H_{min7}$, $H_{min3}$, $H_{min15}$, $H$, $H_{min}$, $W_{max7}$, $W_{max15}$ \\%, $W_{mean7}$, $W_{max3}$, $W_{mean15}$, $W_{mean3}$, $T_{max3}$, $T_{max7}$, $T_{max}$, $W_{max}$, $W_{min7}$, $W_{min15}$ \\ 
PC3: Wind & $W_{mean7}$, $W_{mean3}$, $W_{mean15}$, $W_{max7}$, $W_{max3}$, $W_{max15}$, $W_{min7}$, $W_{min15}$, $W$, $W_{min3}$, $W_{max}$ \\ %, $H_{min7}$, $H_{min15}$, $W_{min}$, $H_{min3}$, $P_{Emean7}$, $D_{t} P_{mean7}$, $D_{t} P_{Emean7}$, $P_{mean7}$, $H_{mean7}$ \\ 
PC4: Precipitation & $P_{mean7}$, $P_{Emean7}$, $P_{mean15}$, $D_{t}:P_{Emean7}$, $P_{Emean15}$, $D_{t}:P_{mean7}$, $P_{mean3}$, $P_{Emean3}$, $D_{t}:P_{Emean15}$, $D_{t}:P_{Emean3}$ \\%, $D_{t} P_{mean15}$, $D_{t} P_{mean3}$, $P$, $P_{eff}$, $D_{t} P_{eff}$, $D_{t} P$, $A_{NDVI}$, $L_{phos}:A_{NDVI}$, $L_{nitro}:A_{NDVI}$, $L_{SLA}:A_{NDVI}$ \\ 
PC5: NDVI & $L_{phos}:A_{NDVI}$, $A_{NDVI}$, $L_{SLA}:A_{NDVI}$, $L_{nitro}:A_{NDVI}$, $V_{Height}:A_{NDVI}$, $G_{mean3}$, $G_{max3}$, $T_{max3}$, $G_{max7}$, $G_{mean7}$ \\ %, $W_{max15}$, $T_{max}$, $W_{max7}$, $H_{min3}$, $T_{mean3}$, $W_{max3}$, $T$, $G_{mean15}$, $G_{max15}$, $S$ \\ 
PC6: Solar Radiation & $G_{max3}$, $G_{max7}$, $G_{mean3}$, $G_{max}$, $G$, $G_{mean7}$, $D_{DC}$, $G_{max15}$, $G_{mean15}$, $B_{UI}$ \\%, $L_{phos}:A_{NDVI}$, $D_{DMC}$, $A_{NDVI}$, $L_{SLA}:A_{NDVI}$, $L_{nitro}:A_{NDVI}$, $V_{Height}:A_{NDVI}$, $W_{min15}$, $S_{elevation}$, $W_{min7}$, $W_{min3}$ \\ 
%PC7 & Precipitation & $P_{eff}$, $P$, $D_{t}:P_{eff}$, $D_{t}:P$, $P_{Emean15}$, $P_{Emean7}$, $P_{mean15}$, $P_{mean7}$, $H_{min}$, $D_{t}:P_{Emean15}$ \\ %, $D_{t} P_{mean15}$, $H$, $P_{mean3}$, $P_{Emean3}$, $D_{t} P_{Emean7}$, $D_{t} P_{mean7}$, $T_{max}$, $G_{max}$, $G$, $G_{max7}$ \\ 
\hline
\end{tabular} 
    \caption{The composition of the first six principal components including the ten most important variables for each component. We have given interpretations of the groupings on the left. See the appendix for a definition of each. Variables which are of the form $X_{agg k}$ where $agg \in \{mean, min, max\}$ and $k \in \mathbb{N}$ indicates the value of $X$ aggregated with it's $k$ previous observations.}
  \label{tab:pca_components} 
\end{table} 

\begin{figure}
    \centering
    \includegraphics[scale=0.7, width=0.5\textwidth]{figures/pca/screeplot.png}
    \caption{The amount of explained variance for each of the first 10 principal components}
    \label{fig:screeplot}
\end{figure}

\subsection{Mixed Effects Modelling} \label{sec:hierarchical}

One strategy to combat the spatial variability that plagues many predictive models of LFMC would be to follow a more mechanistic approach that relies on characteristics of the vegetative species in question. These results could then be connected to the spatial dimension through estimating the distribution of the plants within each region; this portion is outside of the scope of this study. Without a strong model and data that can predict LFMC from some unknown characteristics, we will take these characteristics as latent variables and use a mixed effects model to account for and learn from these variables. We will generate a model both using principal components and using variables selected from Bayesian Model Selection from the general model given by equation: \ref{eqn:lmer_model}. 
\begin{equation}
\begin{aligned}
  \operatorname{LFMC}_{i}  &\sim \mathcal{N} \left(\mu, \sigma^2 \right) \\
  \mu &=\alpha_{j[i]} + \gamma X_{0} +  \beta_{j[i]}X_{1} \\
\left[
  \begin{array}{c} 
    \begin{aligned}
      &\alpha_{j} \\
      &\beta_{j} \\
    \end{aligned}
  \end{array}
\right]
  &\sim \mathcal{N} \left(
\left[
  \begin{array}{c} 
    \begin{aligned}
      &\mu_{\alpha_{j}} \\
      &\mu_{\beta_{j}} \\
    \end{aligned}
  \end{array}
\right]
, 
  \begin{array}{c}
     \Sigma
  \end{array}
 \right)
    \text{, for Species j = 1,} \dots 9
\end{aligned}
\label{eqn:lmer_model}
\end{equation}

LFMC is modelled under a normal distribution with mean $\mu$ and variance $\sigma^{2}$ whose mean parameter is estimated from a mixture of Gaussian distributions composed of three parts. The first part is the intercept for each group given by $\alpha_{j}$. The second part is given by the fixed effects with coefficients $\gamma$ and $X_{0}$ which represents relevant partition of the design matrix. The third part is given by the rest of the design matrix of the predictors used as random effects with group coefficients given by $\beta_{j}$. Finally, $\alpha_{j}$'s and $\beta_{j}$'s are estimated by a normal distribution with mean $\mu_{\alpha_{j} - \beta_{j}}$ and covariance $\Sigma$. \\

For the PCA-based model we use the first twenty two principal components as predictors; these explain 95\% of the cumulative variance. We first fit two baseline: a completely pooled OLS and a completely unpooled OLS. We then fit a model varying the intercepts. In this case in reference to model \ref{eqn:lmer_model} this would mean that $X_{0}$ is the entire design matrix of all predictors. Next we fit a model using all components as random effects; i.e. $X_{1}$ is the entire design matrix. We then fitted a model using just the first six principal components for $X_{1}$ and the other 16 in $X_{0}$; we choose these components through a combination of observing the scree plot (fig \ref{fig:screeplot}) and observing that the first six components still explains nearly 80\% of the cumulative variance. After fitting this model, we run ANOVA tests to determine if all the components are truly significant as a random effect. Under its recommendation we fit a final model using only principal components 1, 3, 5, and 6\footnote{This model is referred to as the PC min model in the rest of this paper.}. \\

For the BMS-based model we use the 54 predictors chosen and follow a similar approach. We make similar baseline plots as well as a varying intercept plot. Due to issues of collinearity amongst the random effects, we cannot use all the predictors that we heuristically chose from our analysis of the marginal posterior probabilities of inclusion. We further trim our choice of predictors by considering their covariances and setting a threshold of 0.9. Finally, we fit a model using eleven predictors: $H_{mean7}$, $D_{DC}$, $T_{min15}$, $W_{mean15}$, $D:D_{DMC}$, $D:W_{max}$, $H$, $G_{max3}$, $G_{max}$, and $L_{nitro}:A_{NDVI}$ to compose $X_{1}$. Using the same ANOVA tests, we remove $H$, $G_{max3}$, $G_{max}$, and $L_{nitro}:A_{NDVI}$ for a minimal model of 6 random effects\footnote{}. \\

Comparing the mixed effect models based on the deviance information criterion (DIC) in table \ref{tab:DIC}, is the PCA-based model that uses all the principal components with a DIC of 49,607 and deviance 49,596, but BMS-based model that uses eleven varying slopes is not that much worse with a DIC of 49,783 and deviance 49,634. Although the BMS model uses fewer random effects, it has a higher number of effective parameters implied by the difference between the DIC and deviance. These models also perform similarly in prediction with out of sample $R^{2}$ of approximately 0.68. \\

\begin{table*}[!htbp] \centering 
\begin{tabular}{@{\extracolsep{5pt}}lccc} 
\hline \\[-1.8ex] 
Models & \multicolumn{1}{c}{Deviance} & \multicolumn{1}{c}{DIC} & \multicolumn{1}{c}{OOS $R^{2}$} \\
    \hline \\[-1.8ex]
PCA - Varying Intercept & 51438 & 52437.8 & 0.5463 \\
PCA - Varying Intercept + 22 PCs & 49595.5 & 49606.9 & 0.6846 \\
PCA - Varying Intercept + 6 PCs & 50088.5 & 50089.7 & 0.6696 \\
PCA - Varying Intercept + 4 PCs & 50160.3 & 510158.4 & 0.6582 \\
BMS - Varying Intercept & 50891.5 & 51808.8 & 0.5961 \\
BMS - Varying Intercept + 11 Slopes & 49634.3 & 49783.2 & 0.6880 \\
BMS - Varying Intercept + 6 Slopes & 49690.2 & 49850 & 0.6802 \\
\hline \\[-1.8ex] 
\end{tabular} 
\caption{A comparison of mixed effect models' deviance, DIC and out of sample $R^{2}$. The PCA models indicate the number of first principal components used as random effects, with the exception that '4 PCs' uses principal components 1, 3, 5, and 6 chosen based on ANOVA tests. The simplest BMS model uses: $H_{mean7}$, $D_{DC}$, $T_{min15}$, $W_{mean15}$, $D:D_{DMC}$, $D:W_{max}$; and the larger model adds on: $H$, $G_{max3}$, $G_{max}$, $L_{nitro}:A_{NDVI}$.}
\label{tab:DIC} 
\end{table*} 


We tested the predictive accuracy of our models on a held out sample and measure the overall $R^{2}$ as well as the species specific $R^{2}$. For our mixed effect models we make one thousand bootstrap simulated predictions\footnote{The predictInterval function was used from merTools which differs from the standard arm::sim by incorporating the uncertainty in the variance of the group parameters by making a few draws of these variances while still treating them as 'fixed'. This leads to a higher prediction interval than \texttt{arm::sim} which only incorporates the uncertainty of the fixed effects and the observation level variances and of course is higher than expected confidence intervals of OLS predictions by nature. }. We make simple predictions with 95\% confidence intervals with our OLS models. Figures \ref{fig:PCA_R2} and \ref{fig:BMS_R2} show that our models can both successfully improve model predictive accuracy as well as reduce the variation between groups. These figures outline the different predictive abilities of the different models using the same base predictors as specified above. By grouping the predictions in the same way for each model, stratified by each individual species group, we can see the ability of the mixed effects models to learn information across groups. In figure \ref{fig:PCA_R2} it is evident that the baseline OLS models have a relatively high dispersion of predictive abilities across species, and the mixed effects models are able to learn information across groups and significantly increase the predictions of species such as Black Sage, Eastern Mojave Buckwheat, Chamise, and Chamise New Growth. These species would likely have been characterized as "low-responding" species by (reference drought study here).  There is a large disparity in the number of observations for each species ranging from a hundred to a few thousand. In this case, we might infer that Eastern Mojave Buckwheat and Black Sage, which had only 206 and 421 observations, benefited in this regard by taking advantage of the grand mean of all groups and pooling towards it. But, the same cannot be said of Chamise, and Chamise New Growth, which have 3,251 and 1,073 observations respectively. Together they make up for nearly two thirds of all observations in this sample. For these two groups we can observe a large improvement from the completely pooled OLS and the completely unpooled OLS predictions which indicates that there are species specific traits that are not accounted for by the predictors. Further improvement is only then achieved significantly when random slopes are incorporated, showing that different species react differently to different predictors, and by characterizing this we can more accurately describe the data. This regularization is effective and useful for both large and small group sizes meaning that the improvements in the global predictive ability is not solely derived from regularizing the groups with small sample sizes\footnote{The appendix contains summary tables of the random effects and estimated coefficients to see the precise changes in estimated coefficients between such models as OLS and the different random effects models}. In the BMS-based models (fig \ref{fig:BMS_R2}) a similar story holds for Chamise and Chamise, New Growth, albeit less drastic. \\

\begin{figure}
    \centering
    \includegraphics[width=0.5\textwidth,scale=0.9]{figures/lmer_pca/PCA_R2.png}
    \caption{Predictive accuracy of different models using the first twenty two principal components as predictors.}
    \label{fig:PCA_R2}
\end{figure}

\begin{figure}
    \centering
    \includegraphics[width=0.5\textwidth, scale=0.9]{figures/lmer_BMS/bms_R2.png}
    \caption{Predictive accuracy of different models using predictors selected using Bayesian model selection.}
    \label{fig:BMS_R2}
\end{figure}
Figure \ref{fig:ols_vs_pca} compares the OLS unpooled estimates and our PCA-based model for Black Sage, one of the species that saw the most drastic improvement. We can observe that the prediction intervals are much greater than the OLS confidence intervals. But, in the end, the bootstrapped mean predictions can be seen to be better than the OLS estimates, especially for the values that are below the 79\% threshold. We notice that some of our prediction intervals fall below zero, which indicates an area of improvement on this model since it is impossible for a plant's LFMC to be below zero.  \\

\begin{figure*}
    \centering
    \includegraphics[scale=0.85, height=0.4\textheight]{figures/mixed_effects/sageblack_bms_pca_comparison_sageblack.png}
    \caption{Out of sample predictions of Black Sage using unpooled OLS and mixed effects model using all 22 principal components as predictors. The red points indicate the true values and the blue dotted line indicates LFMC=79\%. Prediction intervals for the mixed effects model and confidence intervals for OLS are shown for the 95\% level.  }
    \label{fig:bms_vs_pca}
\end{figure*}

Comparing the predictions using the best BMS and PCA based models for Chamise in figure \ref{fig:bms_vs_pca}, there is not a huge difference in their predictions nor their predictive intervals. This is further evident in the residual unexplained variance of the observation level of these models; for the PCA model it is 795.2, and for the BMS model it is 809.8. Although the BMS model's predictors could be a bit easier to interpret.  Intuitively the view that a small number of signals are responsible for the large number of predictors that forms the basis of principal components analysis aligns with the types of predictors and the way they are generated in this model. It also then makes sense that the predictors chosen through BMS include predictors in all categories (temperature, precipitation, humidity, solar radiation, NDVI, drought indices), indicating that BMS is able to select these predictors from amongst an environment of collinearity, but these might not necessarily be the true determinants of LFMC. \\

\begin{figure*}
    \centering
    \includegraphics[scale=0.85, height=0.4\textheight]{figures/mixed_effects/sageblack_ols_pca_comparison_chamise.png}
    \caption{Out of sample predictions for the first 100 observations of Chamise using a PCA-based mixed effects model and a BMS-based mixed effects model. The red points indicate the true values and the blue dotted line indicates LFMC=79\%. Prediction intervals for the mixed effects model and confidence intervals for OLS are shown for the 95\% level. }
    \label{fig:ols_vs_pca}
\end{figure*}



\section{DISCUSSION - 0.25 Page}
We learned the following from the Bayesian Model Selection ... % TODO

Mixed effects models here are an effective way to deal with the dispersion of predictive ability amongst groups of plant species. They provide stronger predictive results and also insights into the different responses of different species that determine their LFMC. There are a few ways in which these models can be improved and extended to provide a more complete picture. Firstly, given more computing resources, it would be possible to perform a more rigorous search of our predictor space for the predictors that can capture the most variance between groups. Moreover, it would be an improvement to run hierarchical models to encode prior domain knowledge. This would include a sensible lower bound to LFMC since living plants have a definite lower bound. As seen in the predictions in figure \ref{fig:ols_vs_pca}, this lower bound was violated. Another way to encode prior beliefs would be to employ quantile regression within a mixed effects model. Since this model, is motivated to improve wildfire prediction models and we know that the 79\% threshold of LFMC is a critical threshold (insert reference here to critical levels of LFMC paper), we could use quantile regression to upweight the importance of these lower observations and downweight the heavy tails in the other direction; we are more interested that vegetation is too dry rather than over saturated. This would be a cleaner way than the use of logistic regression in (reference to drought paper).\\

Further work can also be done by gathering more data. Gathering more LFMC data in a significant way may not be feasible, but gathering more accurately matched meteorological and NDVI data is. This would allow the model to be expanded to find other groupings by including more groups of species as well as to have enough observations to consider the time and spatial dimensions. Firstly, if more species could be included, then there are possibilities of using clustering techniques to discover latent clusterings of the species that could be distinct from the current prevalent groupings: by species, as low and high responding species, or by type (i.e. shrub or tree). As we saw in figure \ref{fig:PCA_R2}, the predictability of certain species was not solely due to a low sample size. Finally, with more observations it would be prudent to find ways to learn more from the data via time and spatial dimensions such as through time series techniques and spatial regression. \\

\onecolumn
\section{APPENDIX}
\subsection{Code}
All source code can be found at https://github.com/antotocar34/fmc\textunderscore prediction

The main code for analysis is in https://github.com/antotocar34/fmc\textunderscore prediction/tree/master/code/analysis/main

\subsection{Formulas}
% \subsubsection*{NDVI Downweight}
% Where subscript $t$ denotes the date of observation and $h$ denotes the time lag in days.
% \subsubsection*{Growth Cycle}
% \begin{equation}
% \label{eqn:growth_cycle}
%     D = \cos\left(\frac{2\pi D}{365} - 0.59\right)
% \end{equation}
\subsubsection*{Drought Code}
\begin{equation}
\label{eqn:dc1}
    DC =
    \begin{cases}
        DC_{t-1} + 0.5 \cdot V & \text{for $P \leq 2.8$} \\
        DC_{r_{t}} + -.5 \cdot V & \text{for $P > 2.8$} \\
    \end{cases}
\end{equation}
Where $DC_{t-1}$ is the previous day's value, and if unavailable is set to 15 and V is the potential evapotranspiration calculated by:
\begin{equation}
\label{eqn:dc2}
    V = 0.36 \cdot (T_{12} + 2.8) + L_{f}
\end{equation}
Where $T_{12}$ is the temperature recorded at noon if $T_{12} \geq -2.8$ and otherwise $T_{12} = -2.8$ and $L_{f}$ is a day-length factor given by Table \ref{tab:day_factor} \\

\begin{table}[!h]
    \centering
    \begin{tabular}{c|c|c|c|c|c|c|c|c|c|c|c}
        Jan & Feb & Mar & Apr & May & Jun & Jul & Aug & Sep & Oct & Nov & Dec  \\
        -1.6 & -1.6 & -1.6 & 0.9 & 3.8 & 5.8 & 6.4 & 5.0 & 2.4 & 0.4 & -1.6 & -1.6 \\
    \end{tabular}
    \caption{Day length factor for the drought code}
    \label{tab:day_factor}
\end{table}
 And $DC_{r_{t}}$ is calculated by first calculating effective precipitation $P_{eff}$ using precipitation $P$:
 \begin{equation}
 \label{eqn:dc3}
     P_{eff} = 0.83 \cdot P - 1.27
 \end{equation}
 Then calculating the day's moisture equivalent after rain $Q_{r_{t}}$:
 \begin{equation}
 \label{eqn:dc4}
     Q_{r_{t}} = 800 \cdot e^{\frac{-DC_{t-1}}{400}} + 3.937 \cdot P_{d}
 \end{equation}
 And finally calculating $DC_{r{_t}}$. Note if $DC_{r_{t}} <0$ then $DC_{r_{t}} = 0$:
 \begin{equation}
 \label{eqn:d5}
     DC_{r_{t}} = 400 \cdot \ln{\frac{800}{Q_{r_{t}}}}
 \end{equation}
\subsubsection*{Duff Moisture Code}
\begin{equation}
\label{eqn:dmc1}
    DMC_{t} = 
    \begin{cases}
        DMC_{t-1} + 100 \cdot K & \text{for} P \leq 1.5 \\
        DMC_{r_{t}} + 100 \cdot K & \text{for} P > 1,5
    \end{cases}
\end{equation}
Where $DMC_{t-1}$ is the previous day's DMC or 6 if unavailable, K is the log drying rate calculated by using the temperature recorded at noon, $T_{12}$ if $T_{12} \geq -1.1$ otherwise $T_{12} = -1.1$ and the relative humidity in percent recorded at noon $H_{12}$ and the effective day length $L_{e}$ given by \ref{tab:eff_day}
\begin{equation}
\label{eqn:dmc2}
    K = 1.894 \cdot (T_{12} + 1.1) \cdot (100 - H_{12}) \cdot L_{e} \cdot 10^{-6}
\end{equation}
\begin{table}[!h]
    \centering
    \begin{tabular}{c|c|c|c|c|c|c|c|c|c|c|c}
        Jan & Feb & Mar & Apr & May & Jun & Jul & Aug & Sep & Oct & Nov & Dec  \\
        6.5 & 7.5 & 9.0 & 12.8 & 13.9 & 13.9 & 12.4 & 10.9 & 9.4 & 8.0 & 7.0 & 6.0
    \end{tabular}
    \caption{Effective day length for duff moisture code}
    \label{tab:eff_day}
\end{table}
And $DMC_{r_{t}}$ is calculated by first calculating effective rainfall $P_{e}$ using precipitation P:
\begin{equation}
\label{eqn:dmc3}
    P_{e} = 0.92 \cdot P - 1.27
\end{equation}
Then calculating the duff moisture content $M_{r_{t}}$:
\begin{equation}
\label{eqn:dmc4}
    M_{r_{t}} = 20 + e^{5.6348 - \frac{DMC_{t-1}}{43.43}} + \frac{1000 \cdot P{e}}{48.77 + b \cdot P_{e}}
\end{equation}
Where $b$ is calculated by:
\begin{equation}
\label{eqn:dmc5}
    b = 
    \begin{cases}
        \frac{100}{0.5 + 0.3 \cdot DMC_{t-1}} & \text{, for} DMC_{t-1} \leq 33 \\
        14 - 1.3 \ln(DMC_{t-1}) & \text{, for} 33 < DMC_{t-1} \leq 65 \\
        6.2 \cdot \ln(DMC_{t-1}) - 17.2 & \text{. for} DMC_{t-1} > 65
    \end{cases}
\end{equation}
And finally:
\begin{equation}
\label{eqn:dmc6}
    DMC_{r_{t}} = 244.72 - 43.43 \cdot \ln (M_{r_{t}} - 20) \text{, if} DMC_{r_{t}} < 0 \text{then} DMC_{r_{t}} = 0
\end{equation}
\subsubsection*{Build Up Index}
\begin{equation}
\label{eqn:bui}
    BUI = 
    \begin{cases}
        0.8 \cdot \frac{DMC \cdot DC}{DMC + 0.4 \cdot DC} & \text{for $DMC \leq 0.4 \cdot DC$}\\
        DMC - (1 - \frac{0.8 \cdot DC}{DMC + 0.4 \cdot DC}) \cdot [0.92 + (0.0114 \cdot DC)^{1.7}] & \text{for $DMC > 0.4 \cdot DC$}
    \end{cases}
\end{equation}
\subsubsection{NDVI}

\subsection{Figures}
1) Extra figures showing other models we tried? \\
2) Figure of selecting rho for data preprocessing \\
3) Figure of NDVI sat image for coolness? \\
\subsection{Tables}
1) table of results from drought code literature paper SEE DROUGHT INDICES R2 RESULTS.DOCX \\
\begin{table}[H]
\begin{tabular}{llccccccccc}
       &      &   DC    & &     &    DMC  & &     &    KBDI & & \\
Species & Site & $R^2$ & RMSE & MAE & $R^2$ & RMSE & MAE & $R^2$ & RMSE & MAE \\
Cistus albidus & D06S3 & 0.52 & 17.2 & 13.3 & 0.57 & 16.2 & 12.9 & 0.46 & 18.2 & 13.9 \\
Cistus albidus & D30S2 & 0.47 & 18.1 & 13.9 & 0.42 & 18.9 & 15.4 & 0.36 & 19.9 & 15.2 \\
Cistus albidus & D34S2 & 0.50 & 14.4 & 11.3 & 0.47 & 14.8 & 11.8 & 0.37 & 16.1 & 12.4 \\
Cistus albidus & D83S2 & 0.64 & 12.8 & 9.5 & 0.25 & 18.3 & 15.4 & 0.62 & 13.1 & 9.8 \\
Cistus albidus & D83S3 & 0.39 & 17.7 & 13.9 & 0.18 & 20.4 & 16.7 & 0.41 & 17.4 & 13.6 \\
Cistus mon & D06S2 & 0.45 & 20.2 & 15.8 & 0.50 & 19.4 & 15.3 & 0.41 & 21.0 & 15.7 \\
Cistus mon & D11S2 & 0.32 & 16.0 & 13.0 & 0.29 & 16.3 & 13.1 & 0.24 & 16.8 & 13.5 \\
Cistus mon & D2AS1 & 0.47 & 18.5 & 14.9 & 0.38 & 20.1 & 16.3 & 0.34 & 20.7 & 16.1 \\
Cistus mon & D2BS2 & 0.08 & 23.4 & 18.9 & 0.23 & 21.4 & 17.1 & 0.06 & 23.7 & 19.1 \\
Cistus mon & D2BS3 & 0.18 & 21.1 & 17.2 & 0.42 & 17.7 & 14.4 & 0.19 & 21.0 & 17.0 \\
Cistus mon & D66S1 & 0.15 & 23.8 & 18.7 & 0.47 & 18.8 & 14.7 & 0.04 & 25.4 & 20.3 \\
Cistus mon & D66S2 & 0.29 & 22.5 & 18.8 & 0.38 & 21.1 & 17.5 & 0.22 & 23.6 & 19.4 \\
Cistus mon & D83S1 & 0.48 & 17.4 & 12.8 & 0.27 & 20.5 & 16.4 & 0.44 & 18.0 & 13.1 \\
Cistus mon & D83S3 & 0.42 & 20.0 & 16.0 & 0.27 & 22.5 & 18.6 & 0.43 & 19.9 & 15.7 \\
Erica arborea & D2AS1 & 0.48 & 13.7 & 10.2 & 0.18 & 17.1 & 12.5 & 0.37 & 15.0 & 11.1 \\
Erica arborea & D2AS2 & 0.51 & 11.1 & 8.8 & 0.45 & 11.7 & 8.9 & 0.47 & 11.6 & 9.2 \\
Erica arborea & D2BS2 & 0.34 & 12.1 & 9.0 & 0.17 & 13.6 & 10.0 & 0.33 & 12.2 & 9.2 \\
Erica arborea & D2BS3 & 0.49 & 10.5 & 8.6 & 0.32 & 12.1 & 9.9 & 0.45 & 10.9 & 8.7 \\
Erica arborea & D66S1 & 0.43 & 11.0 & 8.8 & 0.25 & 12.6 & 9.9 & 0.30 & 12.1 & 9.6 \\
Erica arborea & D83S1 & 0.61 & 11.2 & 8.8 & 0.10 & 17.1 & 13.6 & 0.54 & 12.2 & 9.8 \\
Quercus coccifera & D11S1 & 0.29 & 5.8 & 4.6 & 0.05 & 6.7 & 5.3 & 0.24 & 6.0 & 4.7 \\
Quercus coccifera & D13S1 & 0.40 & 6.5 & 4.7 & 0.10 & 8.0 & 6.1 & 0.36 & 6.7 & 4.9 \\
Quercus coccifera & D13S2 & 0.36 & 6.2 & 4.5 & 0.09 & 7.4 & 5.8 & 0.37 & 6.2 & 4.5 \\
Quercus coccifera & D34S2 & 0.42 & 5.7 & 4.3 & 0.06 & 7.3 & 5.8 & 0.47 & 5.5 & 4.2 \\
Quercus coccifera & D84S1 & 0.54 & 4.4 & 3.4 & 0.10 & 6.2 & 4.8 & 0.56 & 4.3 & 3.3 \\
Quercus ilex & D30S2 & 0.40 & 6.2 & 4.8 & 0.12 & 7.5 & 6.0 & 0.37 & 6.3 & 5.0 \\
Quercus ilex & D83S2 & 0.43 & 4.2 & 3.3 & 0.06 & 5.4 & 4.3 & 0.45 & 4.1 & 3.3 \\
Quercus ilex & D84S1 & 0.56 & 4.5 & 3.6 & 0.06 & 6.6 & 5.4 & 0.61 & 4.3 & 3.4 \\
Quercus ilex & D84S2 & 0.51 & 6.6 & 4.7 & 0.23 & 8.2 & 6.6 & 0.42 & 7.1 & 5.1 \\
Rosmarinus off & D06S1 & 0.67 & 16.5 & 13.2 & 0.58 & 18.4 & 14.8 & 0.61 & 17.8 & 14.0 \\
Rosmarinus off & D11S1 & 0.46 & 16.8 & 13.9 & 0.39 & 17.9 & 14.5 & 0.39 & 17.9 & 14.9 \\
Rosmarinus off & D13S1 & 0.41 & 18.7 & 15.3 & 0.50 & 17.1 & 13.6 & 0.26 & 20.9 & 17.0 \\
Rosmarinus off & D13S2 & 0.47 & 14.7 & 11.2 & 0.45 & 15.0 & 12.1 & 0.32 & 16.7 & 12.8 \\
Rosmarinus off & D84S2 & 0.45 & 24.4 & 20.2 & 0.60 & 20.7 & 16.1 & 0.31 & 27.2 & 23.2 \\
\end{tabular}
\end{table}
2) All results from Hierarchical modelling \\
SUMMARY TABLE OF OLS POOLED BMS \\
SUMMARY TABLE OF OLS UNPOOLED BMS\\
SUMMARY TABLEL OF VAR INTERCEPT MODEL BMS\\
SUMMARY TABLE OF VAR SLOPE BMS 11 MODEL \\
TABLE SHOWING RANEF COEFFCIENTS of BMS 11\\

SUMMARY TABLE OF OLS POOLED PCA\\
SUMMARY TABLE OF OLS UNPOOLED PCA\\
SUMMARY TABLE OF VAR INTERCEPT MODEL PCA\\
SUMMARY TABLE OF VAR SLOPE BMS PC 22 MODEL\\ 
TABLE SHOWING RANEF COEFFICIENTS of PC 22\\
\begin{table}[H]
\centering
\begin{tabular}{lrrrr}
\toprule
  & BMA Estimate & 2.5\% & 97.5\% & $p(\cdot \;|\; \bm{\gamma})$\\
\midrule
Station  Elevation & -3.370478 & -5.127001 & -1.613143 & 0.9999536\\
$G$ & 6.350300 & 4.102470 & 8.593709 & 0.9807483\\
$H_{min}$ & 4.750553 & 2.585580 & 7.093371 & 0.9960038\\
Drought COde & -8.478542 & -11.477374 & -6.585258 & 1.0000000\\
\addlinespace
Solar\_Rad\_Max\_MEAN7 & 8.338130 & 0.000000 & 12.918999 & 0.8532256\\
rel\_humid\_min\_MEAN15 & 16.044091 & 9.492875 & 22.604520 & 0.9999944\\
rel\_humid\_MEAN15 & -9.780191 & -16.890643 & -2.822964 & 0.9954583\\
temp\_min\_MEAN15 & -8.613416 & -14.058442 & -3.061096 & 0.9930988\\
evap\_pot\_MEAN15 & 19.613844 & 14.465487 & 24.875580 & 0.9941110\\
\addlinespace
D\_solar\_rad\_avg & 13.535149 & 8.116125 & 19.126382 & 0.9785908\\
D\_temp\_min & 20.573545 & 13.743946 & 33.975686 & 0.9871351\\
D\_evap\_pot & -18.391673 & -25.613727 & -10.672422 & 0.9848089\\
D\_DC & -4.055092 & -6.317401 & -1.682868 & 0.9774863\\
D\_solar\_rad\_max\_MEAN7 & 32.617237 & 22.700637 & 41.218592 & 0.9597782\\
\addlinespace
D\_wind\_speed\_max\_MEAN15 & -26.134160 & -40.647366 & 0.000000 & 0.8690481\\
D\_temp\_max\_MEAN15 & -152.708005 & -204.046987 & -106.004539 & 1.0000000\\
D\_temp\_min\_MEAN15 & -54.003196 & -69.996789 & -38.484842 & 0.9678281\\
D\_temp\_avg\_MEAN15 & 183.904898 & 168.929414 & 209.749593 & 0.9999964\\
ANDVI & -28.178194 & -30.559631 & -25.766596 & 1.0000000\\
\addlinespace
leaf\_nitr\_mass\_ANDVI & 16.067462 & 14.192315 & 17.920441 & 1.0000000\\
leaf\_phos\_ANDVI & 36.670841 & 34.630477 & 38.768370 & 1.0000000\\
plant\_height\_ANDVI & -4.468335 & -5.291190 & -3.646047 & 0.9998859\\
SLA\_ANDVI & -10.777184 & -11.517194 & -10.067203 & 1.0000000\\
\bottomrule
\end{tabular}
    \caption{BMA estimates of variables with high probability.} % TODO write something smarter here.
    \label{tab:unpooled}
\end{table}

Summary tables SEE RESULTS R.ODT => NEED SUMMARY TABLES \\
3) PCA summary results\\

% Table created by stargazer v.5.2.2 by Marek Hlavac, Harvard University. E-mail: hlavac at fas.harvard.edu
% Date and time: Sat, Dec 11, 2021 - 02:05:03 PM
\begin{table}[!htbp] \centering 
  \caption{} 
  \label{} 
\begin{tabular}{@{\extracolsep{5pt}}lccccccc} 
\\[-1.8ex]\hline 
\hline \\[-1.8ex] 
Statistic & \multicolumn{1}{c}{N} & \multicolumn{1}{c}{Mean} & \multicolumn{1}{c}{St. Dev.} & \multicolumn{1}{c}{Min} & \multicolumn{1}{c}{Pctl(25)} & \multicolumn{1}{c}{Pctl(75)} & \multicolumn{1}{c}{Max} \\ 
\hline \\[-1.8ex] 
PC1 & 3 & 2.877 & 4.120 & 0.498 & 0.498 & 4.066 & 7.633 \\ 
PC2 & 3 & 1.348 & 1.755 & 0.096 & 0.345 & 1.974 & 3.354 \\ 
PC3 & 3 & 1.112 & 1.343 & 0.059 & 0.356 & 1.639 & 2.625 \\ 
PC4 & 3 & 1.119 & 1.314 & 0.057 & 0.384 & 1.649 & 2.588 \\ 
PC5 & 3 & 0.902 & 0.960 & 0.032 & 0.387 & 1.337 & 1.932 \\ 
PC6 & 3 & 0.895 & 0.932 & 0.030 & 0.401 & 1.327 & 1.881 \\ 
PC7 & 3 & 0.872 & 0.884 & 0.027 & 0.414 & 1.295 & 1.790 \\ 
PC8 & 3 & 0.783 & 0.746 & 0.020 & 0.419 & 1.165 & 1.510 \\ 
PC9 & 3 & 0.779 & 0.732 & 0.019 & 0.428 & 1.159 & 1.480 \\ 
PC10 & 3 & 0.753 & 0.692 & 0.016 & 0.436 & 1.122 & 1.389 \\ 
PC11 & 3 & 0.714 & 0.638 & 0.014 & 0.441 & 1.064 & 1.261 \\ 
PC12 & 3 & 0.714 & 0.634 & 0.013 & 0.447 & 1.064 & 1.247 \\ 
PC13 & 3 & 0.704 & 0.619 & 0.012 & 0.453 & 1.050 & 1.206 \\ 
PC14 & 3 & 0.659 & 0.568 & 0.010 & 0.457 & 0.984 & 1.064 \\ 
PC15 & 3 & 0.651 & 0.559 & 0.009 & 0.461 & 0.972 & 1.031 \\ 
PC16 & 3 & 0.638 & 0.547 & 0.008 & 0.465 & 0.954 & 0.986 \\ 
PC17 & 3 & 0.615 & 0.526 & 0.007 & 0.458 & 0.919 & 0.928 \\ 
PC18 & 3 & 0.600 & 0.516 & 0.006 & 0.433 & 0.897 & 0.934 \\ 
PC19 & 3 & 0.589 & 0.509 & 0.006 & 0.413 & 0.880 & 0.940 \\ 
PC20 & 3 & 0.569 & 0.497 & 0.005 & 0.381 & 0.851 & 0.945 \\ 
PC21 & 3 & 0.564 & 0.496 & 0.005 & 0.372 & 0.844 & 0.950 \\ 
PC22 & 3 & 0.560 & 0.495 & 0.004 & 0.363 & 0.838 & 0.954 \\ 
PC23 & 3 & 0.535 & 0.486 & 0.004 & 0.324 & 0.801 & 0.958 \\ 
PC24 & 3 & 0.530 & 0.486 & 0.003 & 0.314 & 0.793 & 0.961 \\ 
PC25 & 3 & 0.521 & 0.485 & 0.003 & 0.300 & 0.780 & 0.964 \\ 
PC26 & 3 & 0.513 & 0.484 & 0.003 & 0.286 & 0.768 & 0.967 \\ 
PC27 & 3 & 0.500 & 0.484 & 0.002 & 0.265 & 0.748 & 0.969 \\ 
PC28 & 3 & 0.485 & 0.485 & 0.002 & 0.242 & 0.727 & 0.971 \\ 
PC29 & 3 & 0.483 & 0.486 & 0.002 & 0.237 & 0.723 & 0.973 \\ 
PC30 & 3 & 0.479 & 0.487 & 0.002 & 0.231 & 0.717 & 0.975 \\ 
PC31 & 3 & 0.475 & 0.488 & 0.002 & 0.224 & 0.711 & 0.977 \\ 
PC32 & 3 & 0.469 & 0.490 & 0.002 & 0.214 & 0.703 & 0.978 \\ 
PC33 & 3 & 0.458 & 0.492 & 0.001 & 0.198 & 0.687 & 0.980 \\ 
PC34 & 3 & 0.456 & 0.494 & 0.001 & 0.193 & 0.683 & 0.981 \\ 
PC35 & 3 & 0.453 & 0.495 & 0.001 & 0.189 & 0.679 & 0.982 \\ 
PC36 & 3 & 0.451 & 0.496 & 0.001 & 0.185 & 0.676 & 0.983 \\ 
PC37 & 3 & 0.447 & 0.498 & 0.001 & 0.178 & 0.669 & 0.984 \\ 
PC38 & 3 & 0.444 & 0.499 & 0.001 & 0.173 & 0.665 & 0.985 \\ 
PC39 & 3 & 0.440 & 0.501 & 0.001 & 0.167 & 0.659 & 0.986 \\ 
PC40 & 3 & 0.436 & 0.503 & 0.001 & 0.161 & 0.654 & 0.987 \\ 
PC41 & 3 & 0.435 & 0.504 & 0.001 & 0.159 & 0.652 & 0.988 \\ 
PC42 & 3 & 0.431 & 0.506 & 0.001 & 0.152 & 0.646 & 0.989 \\ 
PC43 & 3 & 0.430 & 0.507 & 0.001 & 0.150 & 0.644 & 0.989 \\ 
PC44 & 3 & 0.426 & 0.509 & 0.001 & 0.144 & 0.639 & 0.990 \\ 
PC45 & 3 & 0.424 & 0.510 & 0.001 & 0.141 & 0.636 & 0.991 \\ 
PC46 & 3 & 0.420 & 0.512 & 0.001 & 0.135 & 0.630 & 0.991 \\ 
PC47 & 3 & 0.418 & 0.514 & 0.001 & 0.131 & 0.627 & 0.992 \\ 
PC48 & 3 & 0.415 & 0.516 & 0.001 & 0.127 & 0.623 & 0.993 \\ 
PC49 & 3 & 0.411 & 0.518 & 0.0005 & 0.120 & 0.616 & 0.993 \\ 
PC50 & 3 & 0.410 & 0.519 & 0.0005 & 0.117 & 0.614 & 0.994 \\ 
PC51 & 3 & 0.409 & 0.520 & 0.0005 & 0.117 & 0.614 & 0.994 \\ 
PC52 & 3 & 0.406 & 0.522 & 0.0004 & 0.111 & 0.608 & 0.994 \\ 
PC53 & 3 & 0.404 & 0.523 & 0.0004 & 0.108 & 0.605 & 0.995 \\ 
PC54 & 3 & 0.401 & 0.525 & 0.0004 & 0.104 & 0.601 & 0.995 \\ 
PC55 & 3 & 0.399 & 0.527 & 0.0003 & 0.100 & 0.598 & 0.996 \\ 
PC56 & 3 & 0.396 & 0.528 & 0.0003 & 0.096 & 0.594 & 0.996 \\ 
PC57 & 3 & 0.394 & 0.529 & 0.0003 & 0.093 & 0.591 & 0.996 \\ 
PC58 & 3 & 0.394 & 0.530 & 0.0003 & 0.092 & 0.590 & 0.996 \\ 
PC59 & 3 & 0.393 & 0.531 & 0.0003 & 0.091 & 0.589 & 0.997 \\ 
PC60 & 3 & 0.391 & 0.532 & 0.0003 & 0.088 & 0.587 & 0.997 \\ 
PC61 & 3 & 0.389 & 0.534 & 0.0002 & 0.084 & 0.583 & 0.997 \\ 
PC62 & 3 & 0.385 & 0.536 & 0.0002 & 0.079 & 0.578 & 0.997 \\ 
PC63 & 3 & 0.385 & 0.537 & 0.0002 & 0.078 & 0.577 & 0.998 \\ 
PC64 & 3 & 0.381 & 0.539 & 0.0002 & 0.073 & 0.572 & 0.998 \\ 
PC65 & 3 & 0.378 & 0.541 & 0.0002 & 0.068 & 0.567 & 0.998 \\ 
PC66 & 3 & 0.376 & 0.543 & 0.0001 & 0.065 & 0.564 & 0.998 \\ 
PC67 & 3 & 0.374 & 0.544 & 0.0001 & 0.061 & 0.561 & 0.998 \\ 
PC68 & 3 & 0.373 & 0.545 & 0.0001 & 0.060 & 0.559 & 0.998 \\ 
PC69 & 3 & 0.371 & 0.547 & 0.0001 & 0.057 & 0.556 & 0.999 \\ 
PC70 & 3 & 0.369 & 0.548 & 0.0001 & 0.054 & 0.553 & 0.999 \\ 
PC71 & 3 & 0.368 & 0.549 & 0.0001 & 0.052 & 0.551 & 0.999 \\ 
PC72 & 3 & 0.366 & 0.550 & 0.0001 & 0.050 & 0.550 & 0.999 \\ 
PC73 & 3 & 0.366 & 0.550 & 0.0001 & 0.049 & 0.549 & 0.999 \\ 
PC74 & 3 & 0.365 & 0.551 & 0.0001 & 0.048 & 0.547 & 0.999 \\ 
PC75 & 3 & 0.364 & 0.552 & 0.0001 & 0.047 & 0.546 & 0.999 \\ 
PC76 & 3 & 0.363 & 0.553 & 0.0001 & 0.045 & 0.545 & 0.999 \\ 
PC77 & 3 & 0.363 & 0.553 & 0.0001 & 0.045 & 0.544 & 0.999 \\ 
PC78 & 3 & 0.361 & 0.554 & 0.0001 & 0.043 & 0.542 & 0.999 \\ 
PC79 & 3 & 0.361 & 0.554 & 0.0001 & 0.042 & 0.542 & 0.999 \\ 
PC80 & 3 & 0.359 & 0.556 & 0.00005 & 0.039 & 0.539 & 0.999 \\ 
PC81 & 3 & 0.359 & 0.556 & 0.00005 & 0.038 & 0.538 & 0.999 \\ 
PC82 & 3 & 0.358 & 0.557 & 0.00005 & 0.037 & 0.537 & 0.999 \\ 
PC83 & 3 & 0.358 & 0.557 & 0.00005 & 0.037 & 0.536 & 1.000 \\ 
PC84 & 3 & 0.357 & 0.558 & 0.00004 & 0.035 & 0.535 & 1.000 \\ 
PC85 & 3 & 0.356 & 0.559 & 0.00004 & 0.034 & 0.534 & 1.000 \\ 
PC86 & 3 & 0.355 & 0.559 & 0.00004 & 0.033 & 0.533 & 1.000 \\ 
PC87 & 3 & 0.354 & 0.560 & 0.00003 & 0.031 & 0.531 & 1.000 \\ 
PC88 & 3 & 0.354 & 0.560 & 0.00003 & 0.031 & 0.531 & 1.000 \\ 
PC89 & 3 & 0.353 & 0.561 & 0.00003 & 0.029 & 0.529 & 1.000 \\ 
PC90 & 3 & 0.352 & 0.561 & 0.00003 & 0.029 & 0.528 & 1.000 \\ 
PC91 & 3 & 0.351 & 0.562 & 0.00003 & 0.027 & 0.527 & 1.000 \\ 
PC92 & 3 & 0.351 & 0.563 & 0.00002 & 0.027 & 0.526 & 1.000 \\ 
PC93 & 3 & 0.351 & 0.563 & 0.00002 & 0.026 & 0.526 & 1.000 \\ 
PC94 & 3 & 0.350 & 0.563 & 0.00002 & 0.025 & 0.525 & 1.000 \\ 
PC95 & 3 & 0.349 & 0.565 & 0.00002 & 0.023 & 0.523 & 1.000 \\ 
PC96 & 3 & 0.347 & 0.565 & 0.00002 & 0.021 & 0.521 & 1.000 \\ 
PC97 & 3 & 0.347 & 0.566 & 0.00001 & 0.021 & 0.521 & 1.000 \\ 
PC98 & 3 & 0.346 & 0.566 & 0.00001 & 0.020 & 0.520 & 1.000 \\ 
PC99 & 3 & 0.346 & 0.567 & 0.00001 & 0.019 & 0.519 & 1.000 \\ 
PC100 & 3 & 0.345 & 0.567 & 0.00001 & 0.018 & 0.518 & 1.000 \\ 
PC101 & 3 & 0.345 & 0.567 & 0.00001 & 0.018 & 0.517 & 1.000 \\ 
PC102 & 3 & 0.344 & 0.568 & 0.00001 & 0.016 & 0.516 & 1.000 \\ 
PC103 & 3 & 0.344 & 0.568 & 0.00001 & 0.016 & 0.516 & 1.000 \\ 
PC104 & 3 & 0.344 & 0.569 & 0.00001 & 0.015 & 0.515 & 1.000 \\ 
PC105 & 3 & 0.343 & 0.569 & 0.00001 & 0.015 & 0.515 & 1.000 \\ 
PC106 & 3 & 0.342 & 0.570 & 0.00001 & 0.013 & 0.513 & 1.000 \\ 
PC107 & 3 & 0.342 & 0.570 & 0.00001 & 0.013 & 0.513 & 1.000 \\ 
PC108 & 3 & 0.341 & 0.570 & 0.00001 & 0.012 & 0.512 & 1.000 \\ 
PC109 & 3 & 0.341 & 0.571 & 0.000 & 0.011 & 0.511 & 1.000 \\ 
PC110 & 3 & 0.340 & 0.572 & 0.000 & 0.010 & 0.510 & 1.000 \\ 
PC111 & 3 & 0.340 & 0.572 & 0.000 & 0.010 & 0.510 & 1.000 \\ 
PC112 & 3 & 0.339 & 0.572 & 0.000 & 0.009 & 0.509 & 1.000 \\ 
PC113 & 3 & 0.339 & 0.573 & 0.000 & 0.008 & 0.508 & 1.000 \\ 
PC114 & 3 & 0.338 & 0.573 & 0.000 & 0.007 & 0.507 & 1.000 \\ 
PC115 & 3 & 0.338 & 0.574 & 0.000 & 0.006 & 0.506 & 1.000 \\ 
PC116 & 3 & 0.336 & 0.575 & 0.000 & 0.004 & 0.504 & 1.000 \\ 
PC117 & 3 & 0.335 & 0.576 & 0.000 & 0.002 & 0.502 & 1.000 \\ 
\hline \\[-1.8ex] 
\end{tabular} 
\end{table} 

4) Coefficient estimates of best tables\\

% Table created by stargazer v.5.2.2 by Marek Hlavac, Harvard University. E-mail: hlavac at fas.harvard.edu
% Date and time: Thu, Dec 16, 2021 - 11:00:54 AM
\begin{table}[H] \centering 
  \caption{} 
  \label{} 
\begin{tabular}{@{\extracolsep{5pt}} ccccccc} 
\\[-1.8ex]\hline 
\hline \\[-1.8ex] 
 & Fuels & OLS Pooled & OLS Unpooled & Intercept Only & BMS Vars 1 & BMS Vars 2 \\ 
\hline \\[-1.8ex] 
1 & buckwheat, eastern mojave & $0.464$ & $0.455$ & $0.439$ & $0.444$ & $0.381$ \\ 
2 & ceanothus, bigpod & $0.597$ & $0.668$ & $0.646$ & $0.685$ & $0.704$ \\ 
3 & ceanothus, hoaryleaf & $0.572$ & $0.710$ & $0.726$ & $0.712$ & $0.677$ \\ 
4 & chamise & $0.310$ & $0.365$ & $0.364$ & $0.547$ & $0.538$ \\ 
5 & chamise, new growth & $0.446$ & $0.469$ & $0.472$ & $0.581$ & $0.567$ \\ 
6 & sage, black & $0.432$ & $0.519$ & $0.518$ & $0.635$ & $0.627$ \\ 
7 & sage, purple & $0.421$ & $0.427$ & $0.428$ & $0.440$ & $0.400$ \\ 
8 & sagebrush, black & $0.517$ & $0.520$ & $0.517$ & $0.639$ & $0.623$ \\ 
9 & sagebrush, california & $0.512$ & $0.516$ & $0.513$ & $0.614$ & $0.616$ \\ 
10 & Total & $0.561$ & $0.597$ & $0.596$ & $0.688$ & $0.680$ \\ 
\hline \\[-1.8ex] 
\end{tabular} 
\end{table} 


% Table created by stargazer v.5.2.2 by Marek Hlavac, Harvard University. E-mail: hlavac at fas.harvard.edu
% Date and time: Mon, Dec 13, 2021 - 08:11:57 PM
\begin{table}[!htbp] \centering 
  \caption{} 
  \label{} 
\begin{tabular}{@{\extracolsep{5pt}}lccccccc} 
\\[-1.8ex]\hline 
\hline \\[-1.8ex] 
Statistic & \multicolumn{1}{c}{N} & \multicolumn{1}{c}{Mean} & \multicolumn{1}{c}{St. Dev.} & \multicolumn{1}{c}{Min} & \multicolumn{1}{c}{Pctl(25)} & \multicolumn{1}{c}{Pctl(75)} & \multicolumn{1}{c}{Max} \\ 
\hline \\[-1.8ex] 
OLS Pooled & 10 & 0.335 & 0.176 & 0.093 & 0.173 & 0.416 & 0.605 \\ 
OLS Unpooled & 10 & 0.472 & 0.139 & 0.306 & 0.402 & 0.525 & 0.755 \\ 
Intercept Only & 10 & 0.470 & 0.141 & 0.293 & 0.397 & 0.523 & 0.752 \\ 
PC 22 & 10 & 0.594 & 0.127 & 0.385 & 0.542 & 0.677 & 0.756 \\ 
PC 6 & 10 & 0.580 & 0.105 & 0.393 & 0.507 & 0.662 & 0.734 \\ 
PC Min & 10 & 0.564 & 0.103 & 0.403 & 0.493 & 0.652 & 0.713 \\ 
\hline \\[-1.8ex] 
\end{tabular} 
\end{table} 

5) Anova tables\\

% Table created by stargazer v.5.2.2 by Marek Hlavac, Harvard University. E-mail: hlavac at fas.harvard.edu
% Date and time: Tue, Dec 14, 2021 - 06:54:52 PM
\begin{table}[!htbp] \centering 
  \caption{} 
  \label{} 
\begin{tabular}{@{\extracolsep{5pt}} ccccccc} 
\\[-1.8ex]\hline 
\hline \\[-1.8ex] 
 & npar & $\log p(\mathbf{y} \;|\; \mathbf{x})$ & AIC & LRT & Df & Pr(\textgreater Chisq) \\ 
\hline \\[-1.8ex] 
\textless none\textgreater  & $134$ & $-\$24,711.260$ & $49,690.530$ &  &  &  \\ 
$H$ & $122$ & $-24,717.170$ & $49,678.340$ & $11.811$ & $12$ & $0.461$ \\ 
$H_{mean7}$  & $122$ & $-24,736.930$ & $49,717.860$ & $51.330$ & $12$ & $0.00000$ \\ 
solar\_rad\_max\_MEAN3  & $122$ & $-24,713.740$ & $49,671.470$ & $4.948$ & $12$ & $0.960$ \\ 
$DC$  & $122$ & $24,848.630$ & $-49,941.270$ & $274.739$ & $12$ & $0$ \\ 
temp\_min\_MEAN15  & $122$ & $-24,733.510$ & $49,711.030$ & $44.500$ & $12$ & $0.00001$ \\ 
wind\_speed\_avg\_MEAN15  & $122$ & $-24,731.650$ & $49,707.290$ & $40.763$ & $12$ & $0.0001$ \\ 
$D \cdot DMC$  & $122$ & $-24,726.060$ & $-49,696.130$ & $29.604$ & $12$ & $0.003$ \\ 
    $L_{nitro} \cdot A_{NDVI}$  & $122$ & $-24,715.640$ & $49,675.280$ & $8.748$ & $12$ & $0.724$ \\ 
$D$\_wind\_speed\_max  & $122$ & $-24,755.600$ & $49,755.210$ & $88.682$ & $12$ & $0$ \\ 
    $G_{max}$  & $122$ & $-24,714.980$ & $49,673.960$ & $7.428$ & $12$ & $0.828$ \\ 
$D \cdot H$  & $122$ & $-24,718.760$ & $49,681.520$ & $14.994$ & $12$ & $0.242$ \\ 
\hline \\[-1.8ex] 
\end{tabular} 
\end{table} 
% X in (1 + rel\_humid + rel\_humid\_MEAN7 + solar\_rad\_max\_MEAN3 + DC + temp\_min\_MEAN15 + wind\_speed\_avg\_MEAN15 + D\_DMC + leaf\_nitr\_mass\_ANDVI + D\_wind\_speed\_max + solar\_rad\_max + D\_rel\_humid\_min \textbar  Fuel)


% Table created by stargazer v.5.2.2 by Marek Hlavac, Harvard University. E-mail: hlavac at fas.harvard.edu
% Date and time: Wed, Dec 15, 2021 - 09:28:06 AM
\begin{table}[!htbp] \centering 
  \caption{} 
  \label{} 
\begin{tabular}{@{\extracolsep{5pt}} ccccccc} 
\\[-1.8ex]\hline 
\hline \\[-1.8ex] 
 & npar & logLik & AIC & LRT & Df & Pr(\textgreater Chisq) \\ 
\hline \\[-1.8ex] 
\textless none\textgreater  & $52$ & $$-$25,043.690$ & $50,191.380$ & $$ & $$ & $$ \\ 
PC1 in (1 + PC1 + PC2 + PC3 + PC4 + PC5 + PC6 \textbar  Fuel) & $45$ & $$-$25,537.270$ & $51,164.540$ & $987.165$ & $7$ & $0$ \\ 
PC2 in (1 + PC1 + PC2 + PC3 + PC4 + PC5 + PC6 \textbar  Fuel) & $45$ & $$-$25,078.580$ & $50,247.160$ & $69.785$ & $7$ & $0$ \\ 
PC3 in (1 + PC1 + PC2 + PC3 + PC4 + PC5 + PC6 \textbar  Fuel) & $45$ & $$-$25,128.690$ & $50,347.380$ & $170.005$ & $7$ & $0$ \\ 
PC4 in (1 + PC1 + PC2 + PC3 + PC4 + PC5 + PC6 \textbar  Fuel) & $45$ & $$-$25,047.710$ & $50,185.420$ & $8.041$ & $7$ & $0.329$ \\ 
PC5 in (1 + PC1 + PC2 + PC3 + PC4 + PC5 + PC6 \textbar  Fuel) & $45$ & $$-$25,069.010$ & $50,228.010$ & $50.636$ & $7$ & $0$ \\ 
PC6 in (1 + PC1 + PC2 + PC3 + PC4 + PC5 + PC6 \textbar  Fuel) & $45$ & $$-$25,091.280$ & $50,272.560$ & $95.180$ & $7$ & $0$ \\ 
\hline \\[-1.8ex] 
\end{tabular} 
\end{table} 


\subsection{Variable Name definitions} \label{varnames}%
\begin{enumerate}
    \item $D_{t}$ - Day of year
    \item $P$ - Accumulated Precipitation
    \item $DC$ - Drought Code
    \item $DMC$ - Duff Moisture Code
    \item $G$ - Solar Radiation
    \item $T$ - Temperature
    \item $H$ - Relative Humidity
    \item $W$ - Wind Speed
    \item $V_{height}$ - Plant Height
    \item $BUI$ - Build up Index
    \item $L_{phos}$ - Phosphorous per dry mass of leaf.
    \item $L_{nitro}$ - Nitrogen per dry mass of leaf.
    \item $SLA$ - Specific Leaf Area.
\end{enumerate}

\newpage
\bibliography{bibliography.bib}
\end{document}
